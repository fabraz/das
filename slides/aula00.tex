\documentclass{beamer}

%% seleções de localização de idoma
\usepackage[portuges]{babel}
\usepackage[utf8]{inputenc}
\usepackage[T1]{fontenc}
\usepackage{graphicx} % Enable the use of graphs
\usepackage{amsfonts} % Enable special mathemtical characters.
\usepackage{amsmath}  % Enable the use  Multi-line equations, compound symbols, etc
\usepackage{booktabs} % More complex tables
\usepackage{hyperref} % Links
\usepackage{subfig}   % Subfigures

\DeclareOptionBeamer{compress}{\beamer@compresstrue}
\ProcessOptionsBeamer

\mode<presentation>

\useoutertheme[footline=authorinstitutetitle]{miniframes}

% A little hack to display frame counter:
\defbeamertemplate*{footline}{Idorobots theme}
{%
  \begin{beamercolorbox}[colsep=1.5pt]{upper separation line foot}
  \end{beamercolorbox}
  \begin{beamercolorbox}[ht=2.5ex,dp=1.125ex,%
    leftskip=.3cm,rightskip=.3cm plus1fil]{author in head/foot}%
    \leavevmode{\usebeamerfont{author in head/foot}\insertshortauthor}%
    \hfill%
    {\usebeamerfont{institute in head/foot}\usebeamercolor[fg]{institute in head/foot}\insertshortinstitute}
  \end{beamercolorbox}%
  \begin{beamercolorbox}[ht=2.5ex,dp=1.125ex,%
    leftskip=.3cm,rightskip=.3cm plus1fil]{title in head/foot}%
    {\usebeamerfont{title in head/foot}\insertshorttitle}%
    \hfill%
    {\usebeamerfont{title in head/foot}\insertframenumber/\inserttotalframenumber}
  \end{beamercolorbox}%
  \begin{beamercolorbox}[colsep=1.5pt]{lower separation line foot}
  \end{beamercolorbox}
}


\useinnertheme{circles}
\usecolortheme{whale}

\definecolor{idocolor}{rgb}{0.2,0.44,0.74}

\setbeamercolor{structure}{fg=idocolor}
\setbeamercolor{titlelike}{parent=structure}
\setbeamercolor{frametitle}{fg=black}
\setbeamercolor{item}{fg=black}

\mode<all>



\begin{document}
\subtitle{Desenvolvimento Avançado de Software (206601)}   
\title{Recepção e Programa}
\author{Fabricio Braz} 
\date{} 

\frame{\titlepage} 

%\frame{\frametitle{Table of contents}\tableofcontents} 

	\begin{frame}
		\frametitle{Confiança}
		\begin{itemize}
			\item Quem sou eu?
			\item Onde quero chegar?
			\item Qual o papel da sua graduação?
		\end{itemize}
	\end{frame}

	\begin{frame}
		\frametitle{Objetivo}
		\begin{itemize}
			\item Introduzi-los à técnicas e conceitos avançados de projeto e codificação de software.
		\end{itemize}
	\end{frame}
	
	
		\begin{frame}
		\frametitle{Conteúdo Programático}
			\begin{itemize}
				\item Refatoração
					\begin{itemize}
						\item contexto, princípios, situações oportunas e o catálogo de refatorações.
					\end{itemize}
				\item Desenvolvimento Orientado a Testes (TDD)
					\begin{itemize}
						\item introdução, TDD vs. teste tradicional e prática de TDD.
					\end{itemize}
				\item Reuso de software
					\begin{itemize}
						\item vantagens e obstáculos, características e fatores de sucesso.
					\end{itemize}
				\item Técnicas de reuso
					\begin{itemize}
						\item padrões de projeto, componentes e modelos de componentes.
					\end{itemize}
				\item Frameworks
					\begin{itemize}
						\item tipos, vantagens e desvantagens, construção de frameworks.
					\end{itemize}
		\end{itemize}
	\end{frame}
	
		\begin{frame}
		\frametitle{Pré-requisitos}
		\begin{itemize}
			\item Orientação a objetos (Java / C++)
			\item Padrões de Projeto (GoF / POSA)
			\item Teste unitário
		\end{itemize}
	\end{frame}

		\begin{frame}
		\frametitle{Método de Ensino}
		\begin{itemize}
			\item 	Os conteúdos são apresentados por aulas expositivas, que podem envolver práticas de laboratório.
A avaliação do progresso do aluno no conteúdo contemplado pela disciplina é obtida a partir de 
provas envolvendo questões teóricas e práticas.

			\item Para ser aprovado na disciplina, o aluno deverá obter avaliação mínima MM e comparecer em pelo 
menos 75\% das aulas.

		\end{itemize}
	\end{frame}
	
	\begin{frame}
	\frametitle{Bibliografia}
\begin{itemize}
	\item Refactoring: Improving the Design of Existing Code. Fowler M., Beck K., Brant, J. 1a
edição, Addison-Wesley, 1999.
	\item Test-Driven Development By Example. Kent Beck. 4a edição, Bookman, 2000.
	\item Padrões de projeto. Gamma, E., Helm, R., Johnson R., Vlissides J. Bookman, 2000.
	\item C.R.U.I.S.E - Component Reuse in Software Engineering. Almeida, E., Álvaro, A.,
Cardoso, V., Mascena J., Burégio, V., Nascimento, L., Lucrédio, D., Meira, S. 1a edi-
ção, Cesar e-books, 2007.

\end{itemize}
\end{frame}


	\begin{frame}
	\frametitle{Programação}
	
\begin{table}[htbp]
\begin{tabular}{|r|r|l|}
\hline
\multicolumn{1}{|c|}{Aula} & \multicolumn{1}{c|}{Data} & Assunto \\ \hline
1 & 11/03/14 & \shortstack[l]{Recepção \& Programa\\Revisão de Orientação a Objetos} \\ \hline
2 & 13/03/14 & Introdução a testes unitários com Junit \\ \hline
3 & 18/03/14 & \shortstack[l]{Introdução à refatoração \\Princípios em refatoração \\Oportunidades em refatoração} \\ \hline
4 & 20/03/14 & Composição de métodos \\ \hline
5 & 25/03/14 & Movendo características entre objetos \\ \hline
6 & 27/03/14 & Organizando dados \\ \hline
7 & 01/04/14 & Simplificando expressões condicionais. \\ \hline
8 & 03/04/14 & Fazendo as chamadas de métodos mais simples. \\ \hline
9 & 08/04/14 & Lidando com generalização \\ \hline


\end{tabular}
\label{}
\end{table}	
	\end{frame}
	\begin{frame}
	\frametitle{Programação}
	
\begin{table}[htbp]
\begin{tabular}{|r|r|l|}
\hline
\multicolumn{1}{|c|}{Aula} & \multicolumn{1}{c|}{Data} & Assunto \\ \hline
10 & 10/04/14 & Avaliação 1 \\ \hline
11 & 15/04/14 & \shortstack[l]{Introdução ao TDD \\TDD vs. Teste tradicional} \\ \hline
12 & 17/04/14 & Conduzindo um TDD. \\ \hline
13 & 22/04/14 & Exercícios de TDD. \\ \hline
14 & 24/04/14 &  \shortstack[l]{Padrões para TDD:\\ TDD Patterns;\\ red bar patterns;} \\ \hline
15 & 29/04/14 &  \shortstack[l]{Padrões para TDD: testing patterns.\\ Green Patterns.} \\ \hline



\end{tabular}
\label{}
\end{table}	
	\end{frame}
	\begin{frame}
	\frametitle{Programação}
	
\begin{table}[htbp]
\begin{tabular}{|r|r|l|}
\hline
\multicolumn{1}{|c|}{Aula} & \multicolumn{1}{c|}{Data} & Assunto \\ \hline
16 & 06/05/14 & \shortstack[l]{Padrões para TDD:\\xUnit Patterns\\Design Patterns}\\ \hline
17 & 08/05/14 & \shortstack[l]{Padrões para TDD:\\Refactoring\\Mastering TDD.} \\ \hline
18 & 13/05/14 & Conclusões sobre TDD; \\ \hline
19 & 15/05/14 & Avaliação 2 \\ \hline
20 & 20/05/14 & Introdução ao reuso de software \\ \hline
21 & 22/05/14 & \shortstack[l]{Técnicas de reuso de software\\componentes\\atributos de componentes\\composição de componentes} \\ \hline
22 & 27/05/14 & Frameworks de componentes. \\ \hline

\end{tabular}
\label{}
\end{table}	
	\end{frame}
	\begin{frame}
	\frametitle{Programação}
	
\begin{table}[htbp]
\begin{tabular}{|r|r|l|}
\hline
\multicolumn{1}{|c|}{Aula} & \multicolumn{1}{c|}{Data} & Assunto \\ \hline
23 & 29/05/14 & Taxonomia e modelo de componentes \\ \hline
24 & 03/06/14 & Detalhamento de um framework horizontal \\ \hline
25 & 05/06/14 & Construção de um framework \\ \hline
26 & 10/06/14 & Avaliação 3 \\ \hline
27 & 12/06/14 & Aula de contingência \\ \hline
28 & 17/06/14 & Aula de contingência \\ \hline
29 & 24/06/14 & Aula de contingência \\ \hline
30 & 26/06/14 & Aula de contingência \\ \hline
31 & 01/07/14 & Aula de contingência \\ \hline
32 & 03/07/14 & Aula de contingência \\ \hline
33 & 08/07/14 & Aula de contingência \\ \hline
\end{tabular}
\label{}
\end{table}	
	
	

\end{frame}
	
\begin{frame}
		\frametitle{Suporte Instrucional}
		\begin{itemize}
			\item http://54.232.82.121/moodle
					\begin{itemize}
						\item Comunicados.
						\item Práticas.
						\item Slides e código das aulas.
						\item Trabalhos.
			
						\item Notas
						\end{itemize}		
		\end{itemize}
	\end{frame}
	
	
	\begin{frame}				
		\begin{center}
			\vspace{1cm}
			{ \Huge Obrigado!\\}
			\vspace{0.5cm}
			%{\normalsize\texttt{\url{http://54.232.82.121/moodle}}}
		\end{center}
	\end{frame}	 \end{document}

